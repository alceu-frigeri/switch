%%%==============================================================================
% WinEdt pragmas
% !Mode:: "TeX:EN"
% Default Compile engines:
% !TEX program = pdflatex
% !PDFTeXify ext =  --enable-etex  --restrict-write18
% !PDFLaTeX ext  =  --enable-etex  --restrict-write18
% !BIB program = biber
%%%==============================================================================
%% Copyright 2025-present by Alceu Frigeri
%%
%% This work may be distributed and/or modified under the conditions of
%%
%% * The [LaTeX Project Public License](http://www.latex-project.org/lppl.txt),
%%   version 1.3c (or later), and/or
%% * The [GNU Affero General Public License](https://www.gnu.org/licenses/agpl-3.0.html),
%%   version 3 (or later)
%%
%% This work has the LPPL maintenance status *maintained*.
%%
%% The Current Maintainer of this work is Alceu Frigeri
%%
%% This is version {1.0} {2025/05/11}
%%
%% The list of files that compose this work can be found in the README.md file at
%% https://ctan.org/pkg/mkswitch
%%
%%%==============================================================================
\documentclass[10pt]{article}
\RequirePackage[verbose,a4paper,marginparwidth=27.5mm,top=2.5cm,bottom=1.5cm,hmargin={40mm,20mm},marginparsep=2.5mm,columnsep=10mm,asymmetric]{geometry}
\usepackage{codedescribe}
\RequirePackage[inline]{enumitem}
\SetEnumitemKey{miditemsep}{parsep=0ex,itemsep=0.4ex}

%%\usepackage[american,siunitx,cuteinductors,smartlabels,arrowmos,EFvoltages,betterproportions]{circuitikz}
%%\usetikzlibrary{math}

\usepackage{mkswitch}

\RequirePackage[backend=biber]{biblatex}
\addbibresource{mkswitch.bib}

\RequirePackage[hidelinks,hypertexnames=false]{hyperref}


\begin{document}
\tstitle{
  author={Alceu Frigeri\footnote{\tsverb{https://github.com/alceu-frigeri/mkswitch}}},
  date={\tsdate},
  title={The mkswitch Package\break Version \PkgInfo{mkswitch}{version}}
  }
  

\begin{typesetabstract}
 
This package offers two commands aimed at implementing a switch/case alike command.
\end{typesetabstract}

\tableofcontents

\section{Introduction}
There are many ways of implementing a switch case programming structure. Notably, one can use \tsobj{\str_case:nn} from \tsobj[pkg]{expl3}, or go over a loop using \tsobj{\pdfstrcmp}, or construct an if-then-else tower, etc.

This implements a solution, somewhat based on  \cite{stackexchage:switchcase}, which (besides being simple) has the advantage of being constant time: once the cases are set up, suffice a single (internal) if (\tsobj{\ifcsname}) to select the correct code to be executed. 



\begin{tsremark}
  The implementation creates a \tsobj{\csname} for each case, and it uses (at the end) the primitive \tsobj{\ifcsname} to select the correct case.
\end{tsremark}
\begin{tsremark}
  The coding is done using \tsobj[pkg]{expl3}, just for the sake of making it more ``readable'', in the package comments one can find an implementation using just \TeX\  primitives.
\end{tsremark}

\setnewcodekey {mkswitch} 
  {
    keywd2={switch,new,addcase,mkswitch,addcase},
    emph={TextCase,myCase},
    emph2={CaseAstring},
    codeprefix={},
    resultprefix={}
  }


\section{Commands}
Two set of commands are created, one to be used in a \tsobj[pkg]{expl3} code régime, and another set to be used in a user document.

\subsection{User Document ones}
\begin{codedescribe}{\mkswitch}
\begin{codesyntax}%
\tsobj{\mkswitch} \tsmeta{switch} \tsargs[marg]{default-code}
\end{codesyntax}
It will create a new switch \tsobj[marg]{switch}, which will expects a single argument. In case the argument doesn't corresponds to any defined case, \tsobj[marg]{default-code} will be used. The resulting \tsobj[marg]{switch} command is expandable. This is just an alias to \tsobj{\switch_new:Nn}
\end{codedescribe}
\begin{tsremark}
  \verb|#1| can be used in \tsobj[marg]{default-code}.
\end{tsremark}

\begin{codedescribe}{\addcase}
\begin{codesyntax}%
\tsobj{\addcase} \tsmeta{switch} \tsargs[marg]{case,code}
\end{codesyntax}
It will add a \tsobj[marg]{case} to a previously defined \tsmeta{switch} and associates \tsobj[marg]{code} with it. \tsobj[marg]{case} will be fully expanded at definition time. Once defined one can call \tsverb[verb]{\switch{case}}, which will put said \tsobj[marg]{code} in the input stream.   This is just an alias to \tsobj{\switch_addcase:Nnn}.
\end{codedescribe}

\subsubsection{Example}

First we create a switch, and associate a few (or more) cases. Note the possibility of using an auxiliary (fully expandable) macro/command when defining the cases.
\begin{codestore}[switch02]
\def\CaseAstring{case-A}
\mkswitch \myCase  {I~ don't~ know:~ #1\par}
\addcase  \myCase  {\CaseAstring} {A~ was~ used\par}
\addcase  \myCase  {case-B} {B~ was~ used\par}
\end{codestore}

\begin{codestore}[switch02]
\def\somemacro{case-A}
\def\someothermacro{case-X}

If B, then \myCase{case-B}
If A, then \myCase{case-A}
If X, then \myCase{case-X}

if somemacro: \myCase{\somemacro}
if someothermacro: \myCase{\someothermacro}
\end{codestore}

\tscode*[mkswitch]{switch02}[1]
\tsexec{switch02}[1]

To use the \tsargs[marg]{switch}, one just has to call it with \tsobj[marg]{case} as an argument. Note the possibility of using an auxiliary macro/command (which has to be fully expandable) as a \tsobj[marg]{case}.

\tsdemo[mkswitch]{switch02}[2]



\subsection{Expl3 ones}
\begin{codedescribe}{\switch_new:Nn}
\begin{codesyntax}%
\tsobj{\switch_new:Nn} \tsmeta{switch} \tsargs[marg]{default-code}
\end{codesyntax}
It will create a new switch \tsobj[marg]{switch}, which will expects a single, type n, argument. In case the argument doesn't corresponds to any defined case, \tsobj[marg]{default-code} will be used. The resulting \tsobj[marg]{switch} command is expandable. 
\end{codedescribe}
\begin{tsremark}
  \verb|#1| can be used in \tsobj[marg]{default-code}.
\end{tsremark}

\begin{codedescribe}{\switch_addcase:Nnn}
\begin{codesyntax}%
\tsobj{\switch_addcase:Nn} \tsmeta{switch} \tsargs[marg]{case,code}
\end{codesyntax}
It will add a \tsobj[marg]{case} to a previously defined \tsmeta{switch} and associates \tsobj[marg]{code} with it. \tsobj[marg]{case} will be fully expanded at definition time. Once defined one can call \tsverb[verb]{\switch{case}}, which will put said \tsobj[marg]{code} in the input stream. 
\end{codedescribe}


\subsubsection{Example}

First we create a switch, and associate a few (or more) cases. Note the possibility of using an auxiliary (fully expandable) macro/command when defining the cases.
\begin{codestore}[switch01]
\ExplSyntaxOn
\def\CaseAstring{case-A}
\switch_new:Nn \TextCase       {I~ don't~ know:~ #1\par}
\switch_addcase:Nnn \TextCase  {\CaseAstring} {A~ was~ used\par}
\switch_addcase:Nnn \TextCase  {case-B} {B~ was~ used\par}
\ExplSyntaxOff
\end{codestore}

\begin{codestore}[switch01]
\def\somemacro{case-A}
\def\someothermacro{case-X}

If B, then \TextCase{case-B}
If A, then \TextCase{case-A}
If X, then \TextCase{case-X}

if somemacro: \TextCase{\somemacro}
if someothermacro: \TextCase{\someothermacro}

\end{codestore}

\tscode*[mkswitch]{switch01}[1]
\tsexec{switch01}[1]



To use the \tsargs[marg]{switch}, one just has to call it with \tsobj[marg]{case} as an argument. Note the possibility of using an auxiliary macro/command (which has to be fully expandable) as a \tsobj[marg]{case}.

\tsdemo[mkswitch]{switch01}[2]

%\tscode
%\tsdemo*[emph={draw,none},emph2={path},emph3={name},basicstyle={\scriptsize\ttfamily},numbers=left]{demoA}
%\tsresult

\printbibliography

\end{document} 